\documentclass[letterpaper]{article}
\title{CS2323 Homework 2}
\author{CS18BTECH11001}
\usepackage{float}
\usepackage[legalpaper, lmargin=0.5in, rmargin=0.5in, tmargin=0.8in]{geometry}
\usepackage{amsmath}
\usepackage{amssymb}
\renewcommand{\shapedefault}{\itdefault}
\begin{document}
\begin{large}
\maketitle
\begin{center}
\textit{This document is generated by \LaTeX}
\end{center}
\begin{flushleft}
\begin{enumerate}

\item[Q1. ]
Given, for a floating point number :\\[0.1in]
Total No. of Bits = T\\[0.1in]
No. of Exponent Bits = E\\[0.1in]
\begin{enumerate}

\item[(a) ]\textbf{Normal Number}\\[0.1in]
\underline{Smallest value : }\\[0.1in]
Exponent = $0000....001$\\[0.1in]
$\Rightarrow$ actual exponent = $1 - (2^{E-1}-1)$\\[0.1in]
\qquad\qquad\qquad\qquad\:\!\! = $2 - 2^{E-1}$\\[0.1in]
Fraction = $000....00$\quad$\Rightarrow$ Significand = $1.0$\\[0.2in]
Smallest Value = $1.0\times 2^{2 - 2^{E-1}}$\\[0.2in]
$\boxed{\therefore\ Smallest\ value\ in\ normal\ mode\ = 2^{2 - 2^{E-1}}}$\\[0.2in]
\underline{Largest value : }\\[0.1in]
Exponent = $1111....110$\\[0.1in]
$\Rightarrow$ actual exponent = $2^E - 2 - (2^{E-1}-1)$\\[0.1in]
\qquad\qquad\qquad\qquad\:\!\! = $2^{E-1} - 1$\\[0.1in]
Fraction = $111....11$\quad$\Rightarrow$ Significand = $1+1-2^{E+1-T} = 2-2^{E+1-T}$\\[0.2in]
Largest Value = $(2-2^{E+1-T})\times 2^{2^{E-1}-1}$\\[0.2in]
$\boxed{\therefore\ Largest\ value\ in\ normal\ mode\ = (2-2^{E+1-T})\times 2^{2^{E-1}-1} \approx 2^{2^{E-1}}}$\\[0.2in]
\textbf{Denormal Number}\\[0.1in]
\underline{Smallest value : }\\[0.1in]
Exponent = $0000....000$\\[0.1in]
$\Rightarrow$ actual exponent = $0 - (2^{E-1}-2)$\\[0.1in]
\qquad\qquad\qquad\qquad\:\!\! = $2 - 2^{E-1}$\\[0.1in]
Fraction = $000....01$\quad$\Rightarrow$ Significand = $2^{-(T-E-1)}$\\[0.2in]
Smallest Value = $2^{-(T-E-1)}\times 2^{2 - 2^{E-1}}$\\[0.2in]
$\boxed{\therefore\ Smallest\ value\ in\ Denormal\ mode\ = 2^{E+3-T-2^{E-1}}}$\\[0.2in]
\clearpage
\underline{Largest value : }\\[0.1in]
Exponent = $0000....000$\\[0.1in]
$\Rightarrow$ actual exponent = $0 - (2^{E-1}-2)$\\[0.1in]
\qquad\qquad\qquad\qquad\:\!\! = $2 - 2^{E-1}$\\[0.1in]
Fraction = $111....11$\quad$\Rightarrow$ Significand $0+1-2^{E+1-T} $\\[0.2in]
Largest Value = $(1-2^{E+1-T})\times 2^{2-2^{E-1}}$\\[0.2in]
$\boxed{\therefore\ Largest\ value\ in\ Denormal\ mode\ = (1-2^{E+1-T})\times 2^{2-2^{E-1}}= 2^{2-2^{E-1}}-2^{E+3-T-2^{E-1}}}$\\[0.2in]

\item[(b) ]\textbf{FP16}\\[0.1in]
Total No. of Bits(T) = $16$\\[0.1in]
No. of Exponent Bits(E) = $5$\\[0.1in]
\underline{Smallest Normal Value} = $2^{2 - 2^{E-1}}=2^{-14}$\\[0.1in]
\underline{Largest Normal Value} = $(2-2^{E+1-T})\times 2^{2^{E-1}-1}=2^{16}-2^5\approx2^{16}$\\[0.1in]
\underline{Smallest Denormal Value} = $2^{E+3-T-2^{E-1}}=2^{-24}$\\[0.1in]
\underline{Largest Denormal Value} = $2^{2-2^{E-1}}-2^{E+3-T-2^{E-1}}=2^{-14}-2^{-24}$\\[0.2in]
\textbf{bfloat16}\\[0.1in]
Total No. of Bits(T) = $16$\\[0.1in]
No. of Exponent Bits(E) = $8$\\[0.1in]
\underline{Smallest Normal Value} = $2^{2 - 2^{E-1}}=2^{-126}$\\[0.1in]
\underline{Largest Normal Value} = $(2-2^{E+1-T})\times 2^{2^{E-1}-1}=2^{128}-2^{120}\approx2^{128}$\\[0.1in]
\underline{Smallest Denormal Value} = $2^{E+3-T-2^{E-1}}=2^{-133}$\\[0.1in]
\underline{Largest Denormal Value} = $2^{2-2^{E-1}}-2^{E+3-T-2^{E-1}}=2^{-126}-2^{-133}$\\[0.2in]

\item[(c) ]
\underline{Second Smallest Normal Value :}\\[0.1in]
Exponent = $0000....001$\\[0.1in]
$\Rightarrow$ actual exponent = $1 - (2^{E-1}-1)$\\[0.1in]
\qquad\qquad\qquad\qquad\:\!\! = $2 - 2^{E-1}$\\[0.1in]
Fraction = $000....01$\quad$\Rightarrow$ Significand = $1+2^{-(T-E-1)}$\\[0.1in]
Second Smallest Value = $(1+2^{-(T-E-1)})\times 2^{2 - 2^{E-1}}=2^{2-2^{E-1}}+2^{E+3-T-2^{E-1}}$\\[0.1in]
$\boxed{\therefore\ Second\ Smallest\ Normal\ Value =2^{2-2^{E-1}}+2^{E+3-T-2^{E-1}}}$\\[0.2in]
\textbf{FP16}\\[0.1in]
\underline{Smallest Normal Value} = $2^{2 - 2^{E-1}}=2^{-14}$\\[0.1in]
\underline{Second Smallest Normal Value} = $2^{2-2^{E-1}}+2^{E+3-T-2^{E-1}}=2^{-14}+2^{-24}$\\[0.1in]
So, Difference = $2^{-14}+2^{-24}-2^{-14}=2^{-24} $\\[0.1in]
$\boxed{\therefore\ Difference = 2^{-24}}$\\[0.1in]
\clearpage
\textbf{bfloat16}\\[0.1in]
\underline{Smallest Normal Value} = $2^{2- 2^{E-1}}=2^{-126}$\\[0.1in]
\underline{Second Smallest Normal Value} = $2^{2-2^{E-1}}+2^{E+3-T-2^{E-1}}=2^{-126}+2^{-133}$\\[0.1in]
So, Difference = $2^{-126}+2^{-133}-2^{-126}=2^{-133} $\\[0.1in]
$\boxed{\therefore\ Difference = 2^{-133}}$\\[0.2in]

\item[(d) ]\textbf{FP16}\\[0.1in]
\underline{Pros : }\\[0.1in]
\begin{itemize}
\item FP16 has high precision than bfloat16 due to more mantissa bits
\end{itemize}
\underline{Cons : }\\[0.1in]
\begin{itemize}
\item FP16 has low range compared to bfloat16 due to less exponent bits
\item Conversion between FP16 and FP32 is difficult due to different no. of exponent bits as FP32
\end{itemize}
\textbf{bfloat16}\\[0.1in]
\underline{Pros : }\\[0.1in]
\begin{itemize}
\item bfloat16 has high range compared to FP16 due to more mantissa bits
\item Conversion between bfloat16 and FP32 is simple due same no. of exponent bits as FP32
\end{itemize}
\underline{Cons : }\\[0.1in]
\begin{itemize}
\item bfloat16 has much low precision near 1 than FP16\\[0.2in]
\end{itemize}

\item[(e) ]The relative spacing between two consecutive numbers is more in bfloat when compared to FP16 due to less no. of mantissa bits. So, the Range of bfloat16($2^{129}$) is much larger compared with that of the Range of FP16($2^{17}$).\\[0.2in]
\end{enumerate}

\item[Q2. ]Given,\\[0.1in]
No.of bits in Virtual Address = 48\\[0.1in]
No. of entries = 64\\[0.1in]
Physical memory = $2GB = 2 \times 1024 \times 1024 \times 1024 = 2^{31} B$\\[0.1in] 
$\Rightarrow$ No. of bits in physical Address = $\log_2^{2^{31}} = 31$\\[0.1in]
Page Size = $2KB = 2 \times 1024 = 2^{11}$\\[0.1in]
$\Rightarrow$ No. of page offset bits = $\log_2^{2^{11}} = 11$\\[0.1in]
So,No. of page number bits = $48-11 = 37$\\[0.1in]
\quad No. of frame number bits = $31-11 = 20$\\[0.1in]
w.k.t TLB Size = $No.\ of\ Entries \times (No.\ of\ page\ number\ bits + No.\ of\ frame\ number\ bits) $\\[0.1in]
\qquad\qquad\qquad\:\:\:\!\! = $64 \times (37 + 20)$\\[0.1in]
\qquad\qquad\qquad\:\:\:\!\! = $64 \times 57$\\[0.1in]
\qquad\qquad\qquad\:\:\:\!\! = $3648 Bits$\\[0.1in] 
$\boxed{\therefore TLB\ Size = 3648 Bits}$\\[0.2in]

\item[Q3. ]TLB Coverage(or Reach) = $\Sigma\ (page size) \times (Entries)$\\[0.1in]
\qquad\qquad\qquad\qquad\qquad\quad = $4KB \times 128 + 2MB \times 32 + 2GB \times 8$\\[0.1in]
\qquad\qquad\qquad\qquad\qquad\quad = $512KB\ 64MB\ 16GB$\\[0.1in]
\qquad\qquad\qquad\qquad\qquad\quad = $16,843,264KB$\\[0.1in]
$\boxed{\therefore\ TLB\ Coverage = 16,843,264KB}$
\clearpage

\item[Q4. ]Given,\\[0.1in]
Frame Size = $1KB = 1 \times 1024 = 2^{10}$\\[0.1in]
$\Rightarrow$ No. of Frame offset bits = No. of page offset bits = $\log_2^{2^{10}}=10$\\[0.1in]
No. of bits in the Given Addresses = $8\times 4 = 32$\\[0.1in]
$\Rightarrow$ No. of page number bits = $32-10 = 22$\\[0.1in]
So, for the intra-cycle compaction we have to consider the first 22 bits as Corresponding VPNs\\[0.1in]
\begin{table}[h]
\flushright
\begin{tabular}{|c|c|c|c|}
\hline
\textbf{Hexadecimal VA} & \textbf{Binary VA} & \textbf{Binary VPN} & \textbf{VPN after compaction }\\
\hline
0x4795BA21 & 01000111100101011011101000100001 & 0100011110010101101110  &  0100011110010101101110\\
\hline
0x4795BB21 & 01000111100101011011101100100001 & 0100011110010101101110 & ------------\\
\hline
0x5795BA21 & 01010111100101011011101000100001 & 0101011110010101101110 & 0101011110010101101110\\
\hline
0x4785BA21 & 01000111100001011011101000100001 & 0100011110000101101110 & 0100011110000101101110 \\
\hline
\end{tabular}
\end{table}
$\boxed{\therefore\ Total\ No.\ of\ unique\ accesses\ sent\ to\ TLB = 3}$\\[0.3in]

\item[Q5. ]
\begin{enumerate}
\item[(a)]Yes, The bits $[30,29,28]$ of all the weights in the range $[2^{-13}:2^{-2}]$ are same and equal to $0\ 1\ 1$.\\[0.1in]
\item[(b)]The value of bits $[30,29,28]$ of all the weights in the range $[2^{1}:2^{11}]$ are same and equal to $1\ 0\ 0$.\\[0.2in]
\end{enumerate}

\item[Q6. ]Yes, It's possible to reduce the no. of L1 cache misses\\[0.1in]
\underline{Code} :\\[0.1in]
for(int $i=0;i<N;i$++)\{\\[0.1in]
\quad for(int $j=0;j<N;j$++)\{\\[0.1in]
$\qquad a[i][j]=1/b[i][j] * c[i][j];$\\[0.1in]
$\qquad d[i][j]=a[i][j]+c[i][j];$\\[0.1in]
\quad \}\\[0.1in]\}\\[0.2in]
\item[Q7. ]\underline{Invalidating Snooping Protocol : }\\[0.1in]
Given, In the memory location 'L' value of 5 is stored\\[0.1in]
\begin{table}[h]
\centering
\begin{tabular}{|c|c|c|c|c|c|}
\hline
& & \multicolumn{4}{c|}{Value stored at location L in}\\
\hline
CPU Activity & activity on bus & C1's Cache & C2's Cache & C3's Cache & Memory\\
\hline
C1 reads L & Cache miss for L & 5 & & & 5 \\
\hline
C2 reads L & Cache miss for L & 5 & 5 & & 5 \\
\hline
C2 writes 2 to L & Invalidate for L & & 2 & & 2 \\
\hline
C3 reads L & Cache miss for L & & 2 & 2 & 2 \\
\hline
C3 writes 14 to L & Invalidate for L & & & 14 & 14 \\
\hline
C2 reads L & Cache miss for L & & 14 & 14 & 14 \\
\hline
C1 writes 24 to L & Invalidate for L & 24 & & & 24 \\
\hline
\end{tabular}
\end{table}

\item[Q8. ]Yes, It's possible to improve the TLB efficiency\\[0.1in]
\underline{Code} :\\[0.1in]
int main()\{\\[0.1in]
for(int $i=0;i<24;i$++)\\[0.1in]
cout $<< a[i]+c[i]+b[i]+d[i];$\\[0.1in]
\}\\[0.2in]

\item[Q9. ]\underline{Variables showing Temporal locality } : i,N\\[0.1in]
\underline{Variables showing Spacial locality } : array[N]\\[0.2in]

\item[Q10. ]For the first time when the Person1 runs the program it take more time because the data has to be fetched to the cache which includes compulsory misses. For the Second time the as data is already stored in the cache the compulsory misses will be reduced and time is reduced.\\[0.1in]
\end{enumerate}
\end{flushleft}
\end{large}
\end{document}